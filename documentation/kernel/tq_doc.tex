\section{Task/Thread Queues}
These are an abstraction to linked lists that provide a safe way to
keep a list of processes that must be safely read and safely written
even by the scheduler itself. There are extra restrictions on these
lists that make this safety reasonable. Task queues are used by the main
scheduler system to keep track of tasks.
\subsection{Definitions}

\begin{lstlisting}
typedef struct {
	unsigned flags;
	/* this lock is initialized with NOSCHED (see mutexes). */
	mutex_t lock;
	/* current node. Initialized to null. This points into the linked list */
	struct llistnode *current;
	struct llist tql;
} tqueue_t;
\end{lstlisting}

\subsection{Functions and Macros}

\begin{itemize}
\item \function{tqueue\_t *}{tqueue\_create}{\argument{tqueue\_t*}{tq}, \argument{unsigned}{flags}} \\
Creates a new tqueue and returns a pointer to it. If \texttt{tq} is null, it allocates
one using \method{kmalloc}. Otherwise it uses the passed structure. \texttt{flags} currently has no use.

\item \function{void}{tqueue\_destroy}{\argument{tqueue\_t*}{tq}} \\
Cleans up a tqueue element and \method{kfree}'s it if it was allocated when created.

\item \function{struct llistnode *}{tqueue\_insert}{\argument{tqueue\_t*}{tq}, \argument{void*}{item}} \\
Inserts and element \texttt{item} into the task queue \texttt{tq}. If \texttt{tq->current} is \texttt{null},
then it will be set to the head of the list. This may be the object that was inserted, but it is not
guaranteed to be so.

\item \function{void}{tqueue\_remove}{\argument{tqueue\_t*}{tq}, \argument{struct llistnode*}{node}} \\
Removes the linked list node \texttt{node}. Does not free the item inside the node. If \texttt{tq->current}
was pointing to \texttt{node}, then it is set to \texttt{null} (though this may change).

\item \function{void *}{tqueue\_next}{\argument{tqueue\_t*}{tq}} \\
Returns what could be thought of as the next item in the list after \texttt{tq->current}. This 
functionality will become abstracted such that any data structure may be used - currently it
is the next item in the linked list. If \texttt{tq->current} is \texttt{null}, this sets it to
the head of the list and returns the head's item.

\end{itemize}
\subsection{Internals}

We must be careful not to schedule while inside one of the functions that
access a task queue. The scheduler accesses the task queues, so that could
cause...issues...Therefore there are multiple stages of locks that must be maintained
inside these functions. Each function first disables interrupts, then acquired
the mutex. After it's done, it releases the mutex and re-enables interrupts.
There are two cases to consider here:
\begin{itemize}

\item \textbf{Single-Processor Systems}:
This works because when interrupts are disabled the processor will not
get interrupted, so there will not be a reschedule getting in the way. The interrupts
creates all the safety needed when accessing these data structures.

\item \textbf{Multi-Processor Systems}:
These are more complicated. After interrupts are disabled, all we've done
is made sure that the current processor wont screw around with the task queue; other
processors are still running. Thus the mutex is acquired to prevent other processors
from accessing the code at the same time.
\end{itemize}


The functions must take great care not to call any functions that do any of the following:
\begin{itemize}
\item Change interrupt state
\item Call any locking functions
\item Rescedule for any reason
\item Access any other task queue
\end{itemize}
If a function must do one of these things, it must wait until after it
has properly released the task queue before doing so (for example, the remove function
releases the task queue and only then frees the node).

\subsection{Files}
\begin{itemize}

\item \texttt{include/tqueue.h}
\item \texttt{kernel/tqueue.c}

\end{itemize}

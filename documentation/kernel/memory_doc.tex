\section{Physical Memory Management}
This system divides the physical memory of the system in pages, and
keeps track of which pages are used and which are free.
\subsection{Definitions}
\begin{itemize}
\item \texttt{typedef unsigned int addr\_t}
\item \texttt{pm\_num\_pages}: Number of pages of available memory
\item \texttt{pm\_used\_pages}: Number of pages currently allocated
\end{itemize}
\subsection{Functions and Macros}

\begin{itemize}
\item \function{addr\_t}{pm\_alloc\_page}{} \\
Returns a page from the free page list and removes it from the list.
\item \function{void}{pm\_free\_page}{\argument{addr\_t}{address}} \\
Puts a page spcified by \texttt{address} back in the free page list.
\end{itemize}

\subsection{Internals}
The free page list is really a stack that grows upwards from \texttt{PM\_STACK\_ADDR}
to \texttt{PM\_STACK\_ADDR\_TOP}. The mappings of this area of memory are maintained by
pages passed to \method{pm\_free\_page} should a page of that area need to be mapped.
\subsection{Out of Memory Handler}
The action taken by the physical memory manager when the system is deemed `out of memory'
is specified by the config option \texttt{CONFIG\_OOM\_HANDLER}. The valid values of this are:
\begin{itemize}
\item \texttt{OOM\_KILL} will kill the process that caused the out of memory error.
\item \texttt{OOM\_SLEEP} will block the calling process until memory is available. This option
can easily cause the kernel to lock up.
\item any other value will have the kernel panic on an out of memory error.
\end{itemize}
\subsection{Files}
\begin{itemize}
\item \texttt{arch/\$(ARCH)/kernel/mm/physical.c}
\item \texttt{include/mm.h}
\end{itemize}

\section{Virtual Memory Management}
This system handles virtual memory and mapping virtual address spaces. It also provides
functions for cloning an address space for new processes.
\subsection{Definitions}
\begin{itemize}
\item \texttt{mmu\_ready}: Memory management has been fully initialized and paging is enabled
\item \texttt{MAP\_NOCLEAR}
\end{itemize}

\subsection{Functions and Macros}

\begin{itemize}

\item \function{void}{vm\_switch}{\argument{page\_dir\_t*}{pd}} \\
Takes in the virtual address of a page directory, and switching the processor
over to use this directory.

\item \function {unsigned}{vm\_getmap}{\argument{unsigned}{address}, \argument{unsigned*}{map}} \\
Sets \texttt{*map} to, and returns, the physical address that is mapped to the virtual address specified
by \texttt{address}.

\item \function {unsigned}{vm\_getattrib}{\argument{unsigned}{address}, \argument{unsigned*}{ptr}} \\
Sets \texttt{*map} to, and returns, the attributes of the page associated with the virtual address
specified by \texttt{address}.

\item \function {unsigned}{vm\_setattrib}{\argument{unsigned}{address}, \argument{short}{attr}} \\
Sets the attributes of the page associated with the virtual address specified by \texttt{address}
to \texttt{attr}.

\item \function {page\_dir\_t *}{vm\_clone}{\argument{page\_dir\_t*}{dir}, \argument{char}{cow}} \\
Clones the directory \texttt{dir}, but leaves certain sections of the page directory pointing to the same
locations as \texttt{dir} in order to allow the kernel memory to be mapped uniformly across all page
directories. The \texttt{cow} flag specifies whether to set the new directory to copy on write, 1 for
enable, 0 for disable (this feature is unimplemented).

\item \function {int}{self\_free}{\argument{int}{flags}} \\
Frees all areas of the current address space except for the user-space stack.

\item \function {int}{free\_stack}{} \\
Frees the user-space stack of the current address space.

\item \function {int}{vm\_map}{\argument{unsigned}{virt}, \argument{unsigned}{phys}, \argument{unsigned}{attrib}, \argument{unsigned}{options}} \\
Maps the virtual address specified by \texttt{virt} to physical address \texttt{phys}, assigns the mapping the attributes
specified by \texttt{attrib}, and zeros the page. \texttt{options} are as follows:
\begin{itemize}
\item \texttt{MAP\_NOCLEAR} will not zero the page after the mapping has been completed.
\end{itemize}

\item \function {int}{vm\_unmap\_only}{\argument{unsigned}{virt}} \\
Removes the mapping of the virtual address specified by \texttt{virt}, and
executes \texttt{invlpg} on the address (which flushes the entry in the
translation lookaside buffer for this address. This will not free the
page allocated for the page table that this address resided in.

\item \function {int}{vm\_unmap}{\argument{unsigned}{virt}} \\
Calls \method{vm\_unmap\_only} on \texttt{virt}, and then frees the
physical page that was mapped there.

\item \macro{PAGE\_DIR\_IDX}{page} \\
Calculates the page directory index for \texttt{page} (the page number, calulated
by dividing an address by \texttt{PAGE\_SIZE}).

\item \macro{PAGE\_TABLE\_IDX}{page} \\
Calculates the page table index for \texttt{page} (the page number, calulated
by dividing an address by \texttt{PAGE\_SIZE}).

\item \macro{PAGE\_DIR\_PHYS}{dir} \\
Returns the physical address of the directory \texttt{dir} (a virtual address).

\item \macro{disable\_paging}{} \\
Disables paging.

\item \macro{enable\_paging}{} \\
Enables paging.

\item \macro{flush\_pd}{} \\
Rewrites CR3 (x86) with its current value, which should flush the entire
TLB.

\item \function{void}{map\_if\_not\_mapped}{\argument{addr\_t}{loc}} \\
Checks to see if the virtual address specified by \texttt{loc} is mapped
and if not, maps a new physical page to that address and zeros it.

\item \function{void}{map\_if\_not\_mapped\_noclear}{\argument{addr\_t}{loc}} \\
Same as \method{map\_if\_not\_mapped}, except it does not zero the memory.

\end{itemize}

\subsection{Internals}

Page faults are handled in \texttt{pfault.c}. The system checks if
the address faulted upon is an error, or needs to have a physical page
mapped to it (for example, the task's user-space heap). If it is an error, 
then the page fault handler sends a SIGSEGV to the task if it is in ring3
or panics if it is in ring0.

The layout of the page directory is as follows: 
\begin{itemize}
\item \texttt{0} to \texttt{TOP\_LOWER\_KERNEL}
contains memory for the kernel to use, and contains the identity mapped section
of memory that holds the kernel executable. This area is linked when cloned.

\item \texttt{SOLIB\_RELOC\_START} to \texttt{SOLIB\_RELOC\_END}
contains the area where shared libraries will be loaded (this feature
is unimplemented). 

\item \texttt{EXEC\_MINIMUM} is the lowest location where executables may
be loaded.

\item \texttt{TOP\_USER\_HEAP} is the highest location where the heap may extend to.

\item \texttt{MMF\_PRIV\_START} to \texttt{MMF\_PRIV\_END} is where memory mapped
files that are private to this process.

\item \texttt{TOP\_TASK\_MEM\_EXEC} to \texttt{TOP\_TASK\_MEM} contains the stack
and other structures that are maintained across execve calls.

\item \texttt{MMF\_SHARED\_START} to \texttt{MMF\_SHARE\_END} contains memory mapped
files that are shared among processes. This area is linked when cloned.

\item \texttt{KMALLOC\_ADDR\_START} to \texttt{KMALLOC\_ADDR\_END} contains the
memory that is allowed to be allocated by \method{kmalloc}. This area is linked
when cloned. This area is linked when cloned.

\item \texttt{PM\_STACK\_ADDR} to \texttt{PM\_STACK\_ADDR\_TOP} contains the
physical memory stack.

\end{itemize}

\subsection{Files}
\begin{itemize}

\item \texttt{arch/\$(ARCH)/kernel/mm/clone.c}
\item \texttt{arch/\$(ARCH)/kernel/mm/free.c}
\item \texttt{arch/\$(ARCH)/kernel/mm/page.s}
\item \texttt{arch/\$(ARCH)/kernel/mm/pfault.c}
\item \texttt{arch/\$(ARCH)/kernel/mm/virtual.c}
\item \texttt{arch/\$(ARCH)/kernel/mm/vmm\_map.c}
\item \texttt{arch/\$(ARCH)/kernel/mm/vmm\_unmap.c}

\end{itemize}
